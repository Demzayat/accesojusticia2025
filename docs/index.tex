% Options for packages loaded elsewhere
\PassOptionsToPackage{unicode}{hyperref}
\PassOptionsToPackage{hyphens}{url}
%
\documentclass[
]{article}
\usepackage{amsmath,amssymb}
\usepackage{iftex}
\ifPDFTeX
  \usepackage[T1]{fontenc}
  \usepackage[utf8]{inputenc}
  \usepackage{textcomp} % provide euro and other symbols
\else % if luatex or xetex
  \usepackage{unicode-math} % this also loads fontspec
  \defaultfontfeatures{Scale=MatchLowercase}
  \defaultfontfeatures[\rmfamily]{Ligatures=TeX,Scale=1}
\fi
\usepackage{lmodern}
\ifPDFTeX\else
  % xetex/luatex font selection
\fi
% Use upquote if available, for straight quotes in verbatim environments
\IfFileExists{upquote.sty}{\usepackage{upquote}}{}
\IfFileExists{microtype.sty}{% use microtype if available
  \usepackage[]{microtype}
  \UseMicrotypeSet[protrusion]{basicmath} % disable protrusion for tt fonts
}{}
\makeatletter
\@ifundefined{KOMAClassName}{% if non-KOMA class
  \IfFileExists{parskip.sty}{%
    \usepackage{parskip}
  }{% else
    \setlength{\parindent}{0pt}
    \setlength{\parskip}{6pt plus 2pt minus 1pt}}
}{% if KOMA class
  \KOMAoptions{parskip=half}}
\makeatother
\usepackage{xcolor}
\usepackage[margin=1in]{geometry}
\usepackage{graphicx}
\makeatletter
\def\maxwidth{\ifdim\Gin@nat@width>\linewidth\linewidth\else\Gin@nat@width\fi}
\def\maxheight{\ifdim\Gin@nat@height>\textheight\textheight\else\Gin@nat@height\fi}
\makeatother
% Scale images if necessary, so that they will not overflow the page
% margins by default, and it is still possible to overwrite the defaults
% using explicit options in \includegraphics[width, height, ...]{}
\setkeys{Gin}{width=\maxwidth,height=\maxheight,keepaspectratio}
% Set default figure placement to htbp
\makeatletter
\def\fps@figure{htbp}
\makeatother
\setlength{\emergencystretch}{3em} % prevent overfull lines
\providecommand{\tightlist}{%
  \setlength{\itemsep}{0pt}\setlength{\parskip}{0pt}}
\setcounter{secnumdepth}{-\maxdimen} % remove section numbering
\ifLuaTeX
  \usepackage{selnolig}  % disable illegal ligatures
\fi
\IfFileExists{bookmark.sty}{\usepackage{bookmark}}{\usepackage{hyperref}}
\IfFileExists{xurl.sty}{\usepackage{xurl}}{} % add URL line breaks if available
\urlstyle{same}
\hypersetup{
  pdftitle={Acceso a la Justicia y Litigio Constitucional},
  hidelinks,
  pdfcreator={LaTeX via pandoc}}

\title{Acceso a la Justicia y Litigio Constitucional}
\author{}
\date{\vspace{-2.5em}}

\begin{document}
\maketitle

\hypertarget{programa}{%
\subsection{Programa}\label{programa}}

En este curso analizaremos el derecho de acceder a la justicia, su
fuente constitucional, sus garantías de funcionamiento y sus mayores
problemas y dificultades para su ejercicio.

La visión liberal del derecho da por supuesto que cualquier persona
podrá acceder a la justicia cuando su derecho haya sido afectado,
vulnerado o amenazado. El presupuesto sobre el que descanza nuestro
sistema jurídico es que el Poder Judicial podrá entender y propender a
la solución de cualquier conflicto de derechos. Sin embargo, existen
diversas causas que hacen que los tribunales y la protección judicial no
sea para todas las personas igualmente accesible. Algunas tendrán muchas
menos dificultades y otras deberán vencer muchas más barreras para poder
hacer llegar su reclamo a los jueces.

Existen aún hoy algunas leyes y normativas que hacen que este acceso sea
especialmente dificil para ciertos grupos. En esos casos, la solución
podrá ser --paradójicamente-- vía un proceso constitucional.

Sin embargo, también existen supuestos de grupos con mayores
dificultades para hacer valer sus derechos ya no por una normativa
expresa, sino por diversas prácticas de los operadores judiciales y
estatales en general. Estos casos son más dificiles de identificar,
aunque inciden de un modo igual o de mayor performatividad. Así, muchas
veces puede verse que la respuesta judicial resulta patriarcal,
xenófoba, racista, hegemónica, etc.

En este curso haremos una revisión sobre la normativa constitucional
involucrada en el derecho de acceso a la justicia, tomaremos nota de las
garantías procesales que establece la constitución para su eficacia, y
analizaremos de modo crítico cómo y por qué existen grupos identitarios
de población que suele tener mayores dificulades de acceso a los
tribunales.

\hypertarget{objetivo}{%
\subsection{Objetivo}\label{objetivo}}

El objetivo del curso es discutir y analizar de modo crítico las
diversas barreras que afectan al acceso a la justicia, sus aspectos
centrales, y su dimensión estructural. La idea es poder analizar
problemas jurídicos desde el punto de vista del acceso a la justicia y
decidir estrategias para superarlas.

\hypertarget{metodologuxeda}{%
\subsection{Metodología}\label{metodologuxeda}}

Será un curso con un componente teorico sobre el derecho de acceso a la
justicia y sus garantías de funcionamiento, y práctico en el análisis de
la situación de los problemas que los grupos identitarios tienen en
particular. Este aspecto práctico partirá del análisis de diversas
investigaciones empíricas sobre cuáles son estas dificultades, y la
discusión sobre cómo fue resuelta tanto por los tribunales locales como
regionales o internacionales.

Se espera que les estudiantes puedan leer la bibliografía indicada de
modo previo a cada clase, y que los encuetros funcionen no para repetir
sino para discutir la literatura indicada.

Habrá un único examen presencial al final de la cursada.

\hypertarget{cronograma-y-bibliografuxeda}{%
\subsection{Cronograma y
bibliografía}\label{cronograma-y-bibliografuxeda}}

\hypertarget{semana-01-viernes-1503.-unidad-1.-presentaciuxf3n}{%
\subsubsection{Semana 01, viernes 15/03. Unidad 1.
Presentación}\label{semana-01-viernes-1503.-unidad-1.-presentaciuxf3n}}

Presentación de los objetivos y metodología del curso. Introducción al
derecho de acceso a la justicia. Dimensión instrumental y dimensión
sustantiva del derecho. Movimiento de Acceso a la Justicia.
Reconocimiento constitucional.

\begin{itemize}
\tightlist
\item
  Maurino, Gustavo y Sucunza Matías, ``Acceso a la Justicia'', en
  Gargarella- Guidi \emph{Comentarios de la Constitución de la Nación
  Argentina}, tomo 2, pp 895-929, puntos 1 y 2,
  \href{https://drive.google.com/file/d/1Gj7BF0AtPkcnPp7YI7-M2oU-olQHS_I9/view?usp=sharing}{aca}
\end{itemize}

\hypertarget{semana-02-martes-1903-unidad-1.-estuxe1ndares-internacionales}{%
\subsubsection{Semana 02, martes 19/03 Unidad 1. Estándares
internacionales}\label{semana-02-martes-1903-unidad-1.-estuxe1ndares-internacionales}}

\begin{itemize}
\item
  Maurino, Gustavo y Sucunza Matías, ``Acceso a la Justicia'', en
  Gargarella- Guidi \emph{Comentarios de la Constitución de la Nación
  Argentina}, tomo 2, pp 895-929, puntos 3 y 4,
  \href{https://drive.google.com/file/d/1Gj7BF0AtPkcnPp7YI7-M2oU-olQHS_I9/view?usp=sharing}{aca}.
\item
  Reglas de Brasilia sobre Acceso a la Justicia de las Personas con
  vulnerabilidad con la actualización de 2018,
  \href{http://www.cumbrejudicial.org/comision-de-seguimiento-de-las-reglas-de-brasilia/documentos-comision-de-seguimiento-de-las-reglas-de-brasilia/item/817-cien-reglas-de-brasilia-actualizadas-version-abril-2018-xix-cumbre-judicial-asamblea-plenaria-san-francisco-de-quito}{en
  este link}
\item
  Borrador de Convenio Iberoamericano sobre Acceso a la Justicia
  \href{https://drive.google.com/file/d/1ZvhBoVmS245Mt4_mxBN_GJOdlh1jDEL5/view?usp=sharing}{aca}
\end{itemize}

\hypertarget{semana-02-viernes-2203-unidad-2.-concepto-cluxe1sico-y-estructural}{%
\subsubsection{Semana 02, viernes 22/03 Unidad 2. Concepto clásico y
estructural}\label{semana-02-viernes-2203-unidad-2.-concepto-cluxe1sico-y-estructural}}

Patrocinio letrado. Requerir un patrocinio letrado puede ser difícil en
muchos casos. ¿Es una garantía útil? Ahora, son todes les abogades
iguales? Por qué tener dinero es relevante para el sistema judicial?
Concepciones clásica y estructural del acceso a la justicia.

\begin{itemize}
\item
  Gallanter, ``Por qué los poseedores salen adelante?,
  \href{https://drive.google.com/file/d/1Fj3STLmXxHrJglgGgs5JUtDjXy0DSJ3o/view?usp=sharing}{link}
\item
  Bohmer, ``Igualadores Retóricos'',
  \href{https://drive.google.com/file/d/0B50ljTnhr79kMWQ2OTgzYzMtNGNiNS00YjQxLThjYjYtYzg2YWYyNzc0MTlh/view?usp=sharing\&resourcekey=0-QzTl1yGSNgnHMOI_ICHjEg}{aca}
\end{itemize}

\hypertarget{semana-03-martes-2603.-unidad-3.-la-agenda-estructural---vuxedctimas}{%
\subsubsection{Semana 03, martes 26/03. Unidad 3. La Agenda estructural
-
Víctimas}\label{semana-03-martes-2603.-unidad-3.-la-agenda-estructural---vuxedctimas}}

\begin{itemize}
\item
  CELS, ``El acceso a la justicia como una cuestión de derechos
  humanos'', en Informe Anual 2016,
  \href{https://drive.google.com/file/d/1qH6DsnHctmesy6nje2f5NgCyowQ0PkyM/view?usp=sharing}{acá}
  El acceso a la justicia como una cuestión de derechos humanos
\item
  Piqué, María, ``Los derechos de las víctimas de delitos en nuestra
  Constitución'', disponible
  \href{https://drive.google.com/file/d/1NaUqgIjcAosZx6OCPuJzcdG3sbKrqfaC/view?usp=sharing}{aca}
\item
  Declaración de Karina Gimenez, a partir del minuto 28 hasta el minuto
  35.
\end{itemize}

\hypertarget{semana-03-viernes-2903.}{%
\subsubsection{Semana 03, viernes
29/03.}\label{semana-03-viernes-2903.}}

Feriado No hay Clases.

\hypertarget{semana-04-martes-0204.}{%
\subsubsection{Semana 04, martes 02/04.}\label{semana-04-martes-0204.}}

Feriado No hay Clases

\hypertarget{semana-04-viernes-0504.-clase-suspendida}{%
\subsubsection{Semana 04, viernes 05/04. Clase
suspendida}\label{semana-04-viernes-0504.-clase-suspendida}}

\hypertarget{semana-05-martes-0904.-unidad-3.-la-pruxe1ctica---migrantes}{%
\subsubsection{Semana 05, martes 09/04. Unidad 3. La práctica -
Migrantes}\label{semana-05-martes-0904.-unidad-3.-la-pruxe1ctica---migrantes}}

En esta unidad comenzaremos a ver cómo son afectados algunos grupos en
particular.

\begin{enumerate}
\def\labelenumi{\alph{enumi}.}
\tightlist
\item
  Migrantes Expulsiones. Caso Vanessa Gomez
\end{enumerate}

\begin{itemize}
\tightlist
\item
  Expediente judicial CAF 067668/2018
  \href{https://drive.google.com/file/d/1MszNbvuQgl56ukSCbql4ggAAKyiia5Bs/view?usp=sharing}{retencion},
  \href{https://drive.google.com/file/d/11RSJ0lXuMKwhQy-gVHfODbc-AH1fBE0j/view?usp=sharing}{captura}
  y
  \href{https://drive.google.com/file/d/1xsXyT2y3F7sGL8yhK8kG6GTfhkYi7nVw/view?usp=sharing}{apelacion}
\item
  Oteiza y Zayat, A treinta años de la CDN,
  \href{https://drive.google.com/file/d/1rZbIfh3834THG1K07moFq1Cgp_zmYtsk/view?usp=sharing}{aca}
\end{itemize}

\hypertarget{semana-05-viernes-1204.-unidad-3.-la-pruxe1ctica---situaciuxf3n-de-calle}{%
\subsubsection{Semana 05, viernes 12/04. Unidad 3. La práctica -
Situación de
Calle}\label{semana-05-viernes-1204.-unidad-3.-la-pruxe1ctica---situaciuxf3n-de-calle}}

\begin{enumerate}
\def\labelenumi{\alph{enumi}.}
\setcounter{enumi}{1}
\tightlist
\item
  Usurpaciones y derecho de defensa
\end{enumerate}

\begin{itemize}
\item
  Ricciardi y Zayat, \emph{El derecho de defensa en los casos de
  usurpaciones en la Ciudad de Buenos Aires: un estudio empírico}, en
  Revista Institucional de la Defensa Pública de la CABA N° 1,
  disponible
  \href{https://drive.google.com/file/d/1KeqFSqkZ2aBrHXkRcIaZCVu6AMSaX7zH/view?usp=sharing}{acá}
\item
  Protocolo de Salud Mental,
  \href{https://drive.google.com/file/d/1_1DCvbeTGT8gLp9pAMsCejJeJPs-2mj2/view?usp=sharing}{aca}
\end{itemize}

\hypertarget{semana-06-martes-1604.-unidad-3.-la-pruxe1ctica}{%
\subsubsection{Semana 06, martes 16/04. Unidad 3. La
práctica}\label{semana-06-martes-1604.-unidad-3.-la-pruxe1ctica}}

\begin{enumerate}
\def\labelenumi{\alph{enumi}.}
\setcounter{enumi}{1}
\tightlist
\item
  Personas con discapacidad
\end{enumerate}

\begin{itemize}
\item
  Corte IDH, caso ``Furlan y familiares c.~Argentina'', disponible
  \href{https://www.corteidh.or.cr/docs/casos/articulos/seriec_246_esp.pdf}{aca},
  y un resumen oficial
  \href{https://www.corteidh.or.cr/docs/casos/articulos/resumen_246_esp.pdf}{aca}
\item
  CSJN, ``R. M.J s/Insania'',
  \href{https://drive.google.com/file/d/1c29zw3cQDmwNGuO-GDR15KUq7Xc-UZJb/view?usp=sharing}{aca}
\item
  CSJN, ``Asociación Francesa de Beneficencia''
  \href{https://drive.google.com/file/d/1hVROXu_IFbswD_0UtcBpXPZ0gEbTIiQJ/view?usp=share_link}{aca}
\item
  CSJN, ``Institutos Médicos Antártida s/ quiebra s/ inc. de
  verificación (R.A.F. y L.R.H. de F)''
  \href{https://drive.google.com/file/d/1THtTAOVsLhI8y5mQq5mNTLrt-LFdFc-K/view?usp=sharing}{aca}
\item
  CSJN, Ejecuciones contra el Estado
  \href{https://drive.google.com/file/d/1X2oK8aGhqhadJ7YkQCHktaT49Ypk5udZ/view?usp=share_link}{aca}
\end{itemize}

\hypertarget{semana-06-viernes-1904.-unidad-3.-la-pruxe1ctica--niuxf1eces}{%
\subsubsection{Semana 06, viernes 19/04. Unidad 3. La práctica
-Niñeces}\label{semana-06-viernes-1904.-unidad-3.-la-pruxe1ctica--niuxf1eces}}

\begin{enumerate}
\def\labelenumi{\alph{enumi}.}
\setcounter{enumi}{2}
\tightlist
\item
  Niñeces
\end{enumerate}

\begin{itemize}
\tightlist
\item
  Romina Faerman,
  \href{https://drive.google.com/file/d/1vjWDNx026ng38dl2I4G_ojXGlBQt23TT/view?usp=sharing}{Las
  decisiones de los NNA a la luz del principio de autonomía}
\item
  MPT CABA,
  \href{https://drive.google.com/file/d/1lc3H955K8Wx1-7mdtDAJX--CBBcqXfuX/view?usp=sharing}{Violencia
  contra Niños, Niñas y Adolescentes}, presentado por Romina Faerman (en
  especial capítulo 2)
\end{itemize}

\hypertarget{semana-07-martes-2304.-unidad-3.-la-pruxe1ctica}{%
\subsubsection{Semana 07, martes 23/04. Unidad 3. La
práctica}\label{semana-07-martes-2304.-unidad-3.-la-pruxe1ctica}}

Unidad 3. La práctica - Mujeres

\begin{enumerate}
\def\labelenumi{\alph{enumi}.}
\setcounter{enumi}{4}
\tightlist
\item
  Mujeres
\end{enumerate}

\begin{itemize}
\item
  Inecip, casos de violencia de género,
  \href{https://inecip.org/publicaciones/herramientas-jurisprudenciales-para-el-litigio-con-perspectiva-de-genero/}{aca}.
  Leer la introducción y las fichas 1, 3, 5 y 6.
\item
  CEDAW, Observación General 33, disponible
  \href{https://drive.google.com/file/d/1j3suY4Q3JhS5EzbAvs6_K6XSbQsOQ7D7/view?usp=sharing}{acá}
\item
  Ley de Protección integral conta la Mujer
  \href{http://servicios.infoleg.gob.ar/infolegInternet/anexos/150000-154999/152155/texact.htm}{aca}
\end{itemize}

Dejo por aca el video de \href{https://youtu.be/prUqWHKnUd4}{Rita
Segato}

\begin{itemize}
\item
  Pique y Fernandez Valle, ``La garantía de imparcialidad judicial desde
  la perspectiva de género''
  \href{https://drive.google.com/file/d/1atsNG8fNfxcBruDPtPve0N3AmUf6OA3h/view?usp=sharing}{aca}
\item
  Recusación Luz Aime
  \href{https://drive.google.com/file/d/1ZWe1xva3x_Y0ttVg2CZ4eliKBj-8bbBy/preview}{aca}
\end{itemize}

\hypertarget{semana-07-viernes-2604.-unidad-3.-la-pruxe1ctica---lgbtiq}{%
\subsubsection{Semana 07, viernes 26/04. Unidad 3. La práctica -
LGBTIQ}\label{semana-07-viernes-2604.-unidad-3.-la-pruxe1ctica---lgbtiq}}

\begin{enumerate}
\def\labelenumi{\alph{enumi}.}
\setcounter{enumi}{3}
\tightlist
\item
  LGBT Trans
\end{enumerate}

\begin{itemize}
\item
  Informe sombra CEDAW 2016,
  \href{https://drive.google.com/file/d/1Zv0TtTTzgt5N4vSBEo3F7UjiayBQ2BIx/view?usp=sharing}{aca}
\item
  Violencia institucional contra personas trans,
  \href{https://drive.google.com/file/d/1NWDuVTHVpSSWbHTuvP6WPAy6waXvDyXo/view}{ponencia}
\item
  Caso de Luz Aime,
  \href{http://cosecharoja.org/absolvieron-a-luz-aime/}{aca}
\end{itemize}

Les dejo también el decreto del
\href{http://servicios.infoleg.gob.ar/infolegInternet/anexos/340000-344999/341808/norma.htm}{cupo
laboral trans} y la
\href{http://servicios.infoleg.gob.ar/infolegInternet/anexos/350000-354999/351815/norma.htm}{ley
27.636}

\hypertarget{semana-08-martes-3004.-unidad-3.-la-pruxe1ctica---afro}{%
\subsubsection{Semana 08, martes 30/04. Unidad 3. La práctica -
Afro}\label{semana-08-martes-3004.-unidad-3.-la-pruxe1ctica---afro}}

\begin{itemize}
\tightlist
\item
  Alegato de cierre Ministerio Público Fiscal
  \href{https://youtu.be/ZKH82JYVLgc}{Luz Aime}
\end{itemize}

\begin{enumerate}
\def\labelenumi{\alph{enumi}.}
\setcounter{enumi}{4}
\tightlist
\item
  Africanos y afrodescendientes
\end{enumerate}

\begin{itemize}
\item
  Informe Defensoria CABA,
  \href{https://drive.google.com/file/d/1vIoZ7GBl8fwXbZWhqoCxqOHsHMvP7Dki/view?usp=sharing}{aca}
\item
  Angela Davis, ``Raza, Mujeres y clase'', Violación, racismo y el mito
  del violador negro,
  \href{https://drive.google.com/file/d/1ghSHi12WUBBQlkcBQYiA5--Z_rGMnndc/view?usp=sharing}{capítulo
  11} (o aca el
  \href{https://drive.google.com/file/d/1dKW89DDQlKGgsKIE74Mi7i6pnmdTvDxn/view?usp=sharing}{libro
  entero})
\item
  Grupo de Trabajo sobre Afrodescendientes, visita a
  \href{http://daccess-ods.un.org/access.nsf/Get?Open\&DS=A/HRC/42/59/Add.2\&Lang=S}{Argentina}
\end{itemize}

Acá el video de la charla TED de Kimberle Crenshaw sobre
interseccionalidad

\hypertarget{semana-08-viernes-0305.-unidad-3.-la-pruxe1ctica---pueblos-originarios}{%
\subsubsection{Semana 08, viernes 03/05. Unidad 3. La práctica - Pueblos
originarios}\label{semana-08-viernes-0305.-unidad-3.-la-pruxe1ctica---pueblos-originarios}}

\begin{enumerate}
\def\labelenumi{\alph{enumi}.}
\setcounter{enumi}{5}
\tightlist
\item
  Pueblos originarios
\end{enumerate}

\begin{itemize}
\item
  Comite DH, NFL,
  \href{https://drive.google.com/file/d/1CP_TpEaI6XXZIb3ACuyqa8XFPuvbdR3f/view?usp=sharing}{aca}
\item
  Inadi, Dictamen Colihueque,
  \href{https://drive.google.com/file/d/1_t9jnOgqHc9qw7PyU6nuJS0_VORXtdLj/view?usp=sharing¨}{aca}
\end{itemize}

\hypertarget{semana-09-martes-0705.-unidad-4.-litigio-constitucional}{%
\subsubsection{Semana 09, martes 07/05. Unidad 4. Litigio
Constitucional}\label{semana-09-martes-0705.-unidad-4.-litigio-constitucional}}

Estándares de la Corte Suprema para facilitar el acceso a la justicia:

\begin{itemize}
\item
  CSJN
  \href{https://drive.google.com/file/d/1EcDi8VnbuUNH7DlaRPhTMit0IGLLGn0d/view?usp=sharing}{Halabi}
\item
  CSJN
  \href{https://drive.google.com/file/d/186eN_fSToqx4d_W2fRI2pckZA3a5CWCH/view?usp=sharing}{Padec}
\item
  Amparo colectivo DADSE
\end{itemize}

\hypertarget{semana-09-viernes-1005.-examen-final-presencial}{%
\subsubsection{Semana 09, viernes 10/05. Examen final
presencial}\label{semana-09-viernes-1005.-examen-final-presencial}}

\end{document}
